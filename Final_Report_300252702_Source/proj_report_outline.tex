%% $RCSfile: proj_report_outline.tex,v $
%% $Revision: 1.2 $
%% $Date: 2010/04/23 02:40:16 $
%% $Author: kevin $

\documentclass[11pt
              , a4paper
              , twoside
              , openright
              ]{report}


\usepackage{float} % lets you have non-floating floats
\usepackage{listings}
\usepackage{color}
\usepackage[toc,page]{appendix}
\definecolor{codegreen}{rgb}{0,0.6,0}
\definecolor{codegray}{rgb}{0.5,0.5,0.5}
\definecolor{codepurple}{rgb}{0.58,0,0.82}
\definecolor{backcolour}{rgb}{0.96,0.96,0.95}
\newcommand{\rom}[1]{\uppercase\expandafter{\romannumeral #1\relax}}

\newcommand*\rfrac[2]{{}^{#1}\!/_{#2}}



\usepackage{lipsum}                     % Dummytext
\usepackage{xargs}                      % Use more than one optional parameter in a new commands
\usepackage[pdftex,dvipsnames]{xcolor}  % Coloured text etc.
% 
\usepackage[colorinlistoftodos,prependcaption,textsize=small]{todonotes}
\newcommandx{\unsure}[2][1=]{\todo[linecolor=red,backgroundcolor=red!25,bordercolor=red,#1]{#2}}
\newcommandx{\change}[2][1=]{\todo[linecolor=blue,backgroundcolor=blue!25,bordercolor=blue,#1]{#2}}
\newcommandx{\info}[2][1=]{\todo[linecolor=OliveGreen,backgroundcolor=OliveGreen!25,bordercolor=OliveGreen,#1]{#2}}
\newcommandx{\improvement}[2][1=]{\todo[linecolor=Plum,backgroundcolor=Plum!25,bordercolor=Plum,#1]{#2}}
\newcommandx{\thiswillnotshow}[2][1=]{\todo[disable,#1]{#2}}
%


\usepackage{ntheorem}
\newtheorem{hyp}{Hypothesis}

\makeatletter
\newcounter{subhyp} 
\let\savedc@hyp\c@hyp
\newenvironment{subhyp}
 {%
  \setcounter{subhyp}{0}%
  \stepcounter{hyp}%
  \edef\saved@hyp{\thehyp}% Save the current value of hyp
  \let\c@hyp\c@subhyp     % Now hyp is subhyp
  \renewcommand{\thehyp}{\saved@hyp\alph{hyp}}%
 }
 {}
\newcommand{\normhyp}{%
  \let\c@hyp\savedc@hyp % revert to the old one
  \renewcommand\thehyp{\arabic{hyp}}%
} 
\makeatother

\lstdefinestyle{mystyle}{
    backgroundcolor=\color{backcolour},   
    commentstyle=\color{codegreen},
    keywordstyle=\color{magenta},
    numberstyle=\tiny\color{codegray},
    stringstyle=\color{codepurple},
    basicstyle=\footnotesize,
    breakatwhitespace=false,         
    breaklines=true,                 
    captionpos=b,                    
    keepspaces=true,                 
    numbers=left,                    
    numbersep=5pt,                  
    showspaces=false,                
    showstringspaces=false,
    showtabs=false,                  
    tabsize=2
}
 
\lstset{style=mystyle}
\usepackage{url} % for typesetting urls

\usepackage{amsmath}
\usepackage{xcolor}
\newcommand\todonote[1]{\textcolor{red}{#1}}
%
%  We don't want figures to float so we define
%
\newfloat{fig}{thp}{lof}[chapter]
\floatname{fig}{Figure}

%% These are standard LaTeX definitions for the document
%%                            
\title{Identifying Redundant Test Cases}
\author{Marc Shaw : 300252702}

\definecolor{darkgray}{rgb}{.4,.4,.4}
 
\lstdefinelanguage{AspectJ}[]{Java}{
    morekeywords={declare, pointcut, aspect, before, around, after, returning, throwing, call, execution, this, target, args, within, withincode, get, set, initialization, preinitialization, staticinitialization, handler, adviceexecution, cflow, cflowbelow, if, proceed},
    moredelim=[is][\textcolor{darkgray}]{\%\%}{\%\%},
    moredelim=[il][\textcolor{darkgray}]{§§}
}

%% This file can be used for creating a wide range of reports
%%  across various Schools
%%
%% Set up some things, mostly for the front page, for your specific document
%
% Current options are:
% [ecs|msor]              Which school you are in.
%
% [bschonscomp|mcompsci]  Which degree you are doing
%                          You can also specify any other degree by name
%                          (see below)
% [font|image]            Use a font or an image for the VUW logo
%                          The font option will only work on ECS systems
%
\usepackage[image,ecs,behons]{vuwproject}

% You should specifiy your supervisor here with
\supervisors{David J Pearce and A/Prof. Lindsay Groves}

% Unless you've used the bschonscomp or mcompsci
%  options above use
%   \otherdegree{OTHER DEGREE OR DIPLOMA NAME}
% here to specify degree

% Comment this out if you want the date printed.
\date{}

\begin{document}

% Make the page numbering roman, until after the contents, etc.
\frontmatter

%%%%%%%%%%%%%%%%%%%%%%%%%%%%%%%%%%%%%%%%%%%%%%%%%%%%%%%

%%%%%%%%%%%%%%%%%%%%%%%%%%%%%%%%%%%%%%%%%%%%%%%%%%%%%%%

\begin{abstract}
This project investigates tracing the method call information of Java JUnit Tests and using the information to identify redundant tests in a test suite. There are a variety of different methods implemented (and experimented on) to identify redundancy within a suite. The techniques involve retrieving different information from the tests and analysing it differently. The experiments show two main points. First, that a set of the method calls work well as a heuristic. This allowed for a pipeline approach to be implemented where the output of the heuristic is pipelined into more thorough analysis. Second, they showed that taking into account for the parameter information of the method calls gave insight into the context of the call. This increased the confidence that the identified tests were truly redundant.
\end{abstract}

%%%%%%%%%%%%%%%%%%%%%%%%%%%%%%%%%%%%%%%%%%%%%%%%%%%%%%%



\maketitle

\chapter*{Acknowledgments}\label{C:ack} 
I wish to thank Dr. David J. Pearce for the support throughout the project. This involved providing feedback, advice and general help. Without such, this project would have suffered greatly. I would like to thank Dr. Roman Klapaukh, Micheal Winton and Hai Tran for the encouragement throughout the year and A/Prof Lindsay Groves for helping with some proof reading.

\tableofcontents

% we want a list of the figures we defined
\listof{fig}{Figures}

\listoftables

%%%%%%%%%%%%%%%%%%%%%%%%%%%%%%%%%%%%%%%%%%%%%%%%%%%%%%%

\mainmatter

%%%%%%%%%%%%%%%%%%%%%%%%%%%%%%%%%%%%%%%%%%%%%%%%%%%%%%%

% individual chapters included here
\chapter{Introduction}\label{C:intro}

Test suites are an important part in every major software project \cite{jeffrey2005test}. They check a specified set of behaviours are met by a software program. Meeting the behaviours increases confidence in the program, but as a consequence, a large number of tests need to be executed and may take up to several hours to run.

Test cases are added to the test suite throughout the project, often when a piece of code is altered, new code is developed or a bug gets fixed \cite{issuetrack,whentotest}. With tests being added throughout, we need to ensure that the thousandth test case is not replicating the behaviour of any of the previous. Even with careful planning, it is difficult to have no redundant test cases. A redundant test case is one which is replicating the behaviour of another.  The goal of the report is to create a tool to identify these test cases within a test suite.

The execution of a test case leaves a trail of data. This trail contains information ranging from low level to high level run time data. A low level example would be machine code while a high level could be the method execution data. A variety of previous research \cite{wong1995effect, wong1999test, rothermel1998empirical, rothermel2002empirical,koochakzadeh2009test,zhang2011empirical,li2008static} discuss using some of this information to identify redundant test cases. The papers identify redundant tests using statement coverage while the majority examine benchmarks with under 1,000 test cases. For benchmarks with a large number of test cases storing every statement execution could be costly. Therefore our report investigates using method execution data to identify redundant test cases. One of the issues that previous studies reported were high level of false positives. Method execution data gives a few different ways to explore this issue which are explored further on. 

The tool developed can analyse method execution details, known as \textit{test spectra}, to determine the level of redundancy between two tests. Using this analysed information, potentially redundant test cases can be displayed to the developers. A basic visualisation of the idea is shown in Figure \ref{fig:spectra}, the spectra is three different tests where colours represent method executions. Intuitively, it is clear that Test 1 is different from Test 2 and 3 as the method executions is different between them. However, there are similarities between Test 2 and 3 which could identify some level of redundancy between the test cases. 

\begin{figure}[h]
\centering
\includegraphics[width=6cm,height=3cm]{spectra.png}
\caption{A figurative spectra where each colour represents different method execution details. We see similarities between Test 2 and 3. In contrast, Test 1 is different from 2 and 3. }
\label{fig:spectra}
\end{figure}

Once the tests are identified as being redundant, they may be removed, however it is important to understand the dangers of removing test cases. Unless two test cases are exactly the same, it is difficult to guarantee that they are redundant, even if one subsumes another. This creates the need to be able to use different techniques, dependent on the project. Therefore to expand on the goal of the project, the developed tool should give developers different approaches for identifying redundant test cases. The tool should allow a developer to configure different analysis metrics and view the results in a file. Overall, the tool is useful for gaining an overview and understanding of the condition of the test suite, allowing for manual inspection to determine if the identified redundant tests should be removed.

Another potential use case of the tool is to redistribute the test cases. This can be achieved by splitting the non redundant tests into another test suite and running this suite in place of the original. The original test suite may be run over night when no development is occurring. This separation of tests based on redundancy allows for a testing process, for example regression testing, to occur in a timely fashion while ensuring the original bug finding ability is retained. This is applicable to David Pearce, he is currently writing a language called Whiley. The language contains an extended static checking tool in order to eliminate run time exceptions through formal verification techniques. In the main compiler module alone, there are roughly 20,000 tests. Relocating a number of these tests into another suite would result in allowing him to increase development speed due to a reduction in the time taken to run a large test suite. Increasing the frequency of test suite execution also allows for bugs to be traced back to code changes easier and reduces the time spent debugging. Therefore, by redistributing the test suite, not only does the time taken to run a test suite decrease, there is incentive for developers to execute the suite more often and in turn helping them reduce time taken to debug.

The contributions of the report is split into two sections: 

\begin{itemize}
\item Create a tool for identifying redundant test cases
\item Analyse different strategies for identifying redundant test cases through experimenting on realistic benchmarks
\end{itemize}

\section{Outline}

The report is structured as follows. Chapter 2 discusses background information that explores concepts needed to understand the following chapters as well as previous research in this area of interest. The design and implementation of the tool is then examined in Chapter 3. After the tool is discussed, Chapter 4 investigates the different techniques explored and conducts and analyses experiments. Finally, Chapter 5 concludes the report and identifies future work.
\chapter{Background}\label{C:related}

\section{Continuous Integration}

The idea was first introduced by Grady Booch \cite{booch2006object}. It is described as merging of all the developer work several times a day and to run automated tests to ensure that the any of the changes do not break the project. This leads to reduced amount of integration issues, and if they do occur they would be smaller as the amount worked on is less. However, this means that the testing after an integration has to be done in a timely fashion otherwise it won’t be finished by the time of the next integration.

\section{Test Driven Development}
\todo{Talk about this?}

\section{Related Work}

Testing is a critical part to any software, not only to stop incidents stemming from it, but because of the increase in popularity of agile \cite{chaos} where software development processes such as the two identified previously are often backbones to many of the agile methodologies. Code integration is becoming increasingly important, so is ensuring that the testing of the integration is within a time frame. For this reason there have been papers that examine the different approaches that can be taken to identify redundant test cases and reduce the size of the test suites. Although reducing the size of the test suite programmatically could lead to issues. Unless two tests are exactly the same then there is no guarantee that two tests are redundant, even when one is a direct subset of another. This is because if $A \subset B$, then the state of the program may be different if $A \sqcup B \ne  \emptyset $.\todo{Dont think that last past was worded well.} In this case, the tests may indeed look similar from a programs point of view, but in reality are testing different situations.

There are two sets of approaches used, static and dynamic checking for redundancy. Li, Francis and Robinson \cite{li2008static} examined through static checking the number of test that may be redundant. Through using three different metrics, Manhattan distance, unigram cosine similarity and bigram cosine similarity they concluded that between 7\% - 23\% were redundant and required manual checking \todo{Feel Like I need to look at another dynamic papers range}. The range concluded may show that static detection can be useful in situations where dynamic data can not be gathered. Fraser and Wotawa \cite{fraser2007redundancy} use the coverage of a test suite to determine the redundancy level as they define it containing redundant if a part of a test case does not contribute to the fault detection ability. A execution tree is used, which is similar to a partial context tree. However, they fail to take into account for changes in state of the program that may occur outside of the similarity. So removing a rest is more likely to lead to less bug identification.
\chapter{Work Done}\label{C:workdone}

\section{Overview}
There were three main goals before the preliminary report. Firstly, create a framework to trace a tests spectrum. This not only included writing the aspect but also ability to pipeline the analyse algorithms. Secondly, find suitable bench marks meeting a certain criteria. Lastly, research and implement several different analysis methods.

\section{Creating the Framework}

The first goal of the project was to create a framework to trace tests and allow for different analysis methods to be added and changed with ease. The first question that arose was, what was the best way to execute every test? Since AspectJ with load time weaving was determined to be the best method to trace, a build automation tool would be best suited. Using a build automation tool will remove a lot of redundancy during the analysis and execution of the tests. Its allows for a script to be written that will be run each time from a single command line argument, rather than having to repeat a long winded command line argument every time \todo{Feels broken?}. The two main choices were Ant and Maven. After doing research, Ant appeared to give more control over the execution of JUnit tests \cite{antjunit} \cite{mavenjunit}. It gave the ability to create a single new java virtual machine (JVM) before the tests were executed which would be used solely for the test suite. Since a static class to hold the trace data was used, the data needed to be held in memory until all the tests had run, which meant the tests had to be run with the same JVM. But the JVM had to be recreated as AspectJ load-time weaving requires its own class loader to be used through the command line java option. 

The ability to allow for a user defined pipeline is important in limiting the amount of hard coding. Ant uses it’s own JUnit runner class to execute the tests. This means that parameters would be difficult to input without executing the Ant file through another main method in order to create the pipelines. Doing this would be create messy and unwanted code. Therefore a properties file was the other viable approach which would be loaded in during the static method construction. The purpose of this properties file was to not only a change to the settings of a run, such as the results file name, the type of service it should use but also allowing for a ‘pipeline’ approach to be used. This pipeline will be used for the analysis where the pipeline can specify the type of analysis and its setting, shown in figure \ref{fig:pipeline}. The main pattern used to limit the amount of hard code when parsing the file was factory patterns. \todo{Show class diagram of a pipeline?}

\begin{figure}[h]
\includegraphics[width=\textwidth]{Pipeline.jpg}
\caption{Pipeline}
\label{fig:pipeline}
\end{figure}


\section{Finding Benchmarks}

The second goal was to find suitable benchmarks; this was harder than it seemed. The benchmarks had to meet a criteria where they were java based, had large number of test cases (100 +) and were open source. Although there were over 10 potential benchmarks, the ability to use them depended on their build process. If they used Maven, it was difficult to create a jar that contained the tests and often meant that the amount of effort needed to get a working benchmark was higher than the benefit from it. This eliminated several potential benchmarks and left the Ant and gradle built projects. 

The current set of benchmarks is as follows:

\begin{itemize}
\item Whiley
\item Java Compiler Kit
\item Jasm
\item Spring - Core
\item Metric-x - Core
\item Ant
\end{itemize}


Using Ant as a benchmark created an unique situation. Since Ant is used to run the JUnit tests, I have made my aspect ignore Ant calls so that when Ant is executing the tests it does not get traced. However, if I ignore Ant calls then the Ant benchmark can not be traced. This meant that I had to not use Ant to run itself but manually execute it. This was deemed worth the effort due to Ant’s current stature within the development community. 

\section{Implementing Analysis Methods}

The third goal was to use the data to analyse and determine the tests that are closely related. The first analysis stage that I decided to use was Levenshtein distance metric. This metric is the minimum number of single-character edits, in our case method calls where edits are either insertions, deletions or substitutions. This method created some challenges. Firstly, the time to compare every test with every test becomes exponential. So as the analysis became more computationally heavy, the time to analyse the data increased. Secondly, due to there being thousands of tests and each test containing tens of thousands of method calls, memory management became crucial. 

My first approach was to resolve both issues by using a bloom filter approach. A bloom filter is a test where you determine whether an element is a member of a set or not, returning either "possibly in set" or "definitely not in set" \cite{bloomfilterwiki}. In this case, it was used to determine whether the set of method calls for two tests were similar or not. The similarity and analysis type are set within the property file. Looking at the set of method calls, rather than a list meant the number of comparisons decreased. Using a 99 percent similarity for the 'bloom filter' on the wyc package of Whiley, this meant the next analysis stage had 232 comparisons, rather than 187922. This approach was used under the hypothesis that the following is true.

$A, B, C, D \neq A, D, E, F, G$

$A, B, C, D \approx A, B, C, D, E$

That the first case the two tests would be not be redundant test’s due to having a high difference in method calls. However, there is a chance that the second test may be the same so it implies more computational heavy analysis should be done on it, such as using a list of method calls. This means that when using a list of method calls, the amount of data held in memory will be reduced as the number of tests will be decreased from the initial stage.

Although this made the analysing stage faster, it did not have any effect on the testing time. Each time that an analysis was to be run, the tests needed to be run at the same time. The approach used to get around this was to save the test data to disk. Such that the properties file can be set to save the data to disk, and when the data wants to be used without analysis, a main method is called. However, analysing was still taking a substantial amount of time. \todo{Provide figures?}

To increase the performance of analysis, I decided to implement a concurrent method. The framework was implemented such that it was able to be done relatively easy. When testing the execution on a test environment, the results between concurrent and non concurrent gave the same result. However, on a benchmark the result was different. This required thinking about whether the test environment was too limited. David Pearce suggested a larger test environment be created. Through doing this it allowed the errors to be found and in turn the concurrent execution was fixed. The concurrent execution lead to an increase of up to 2 times faster analysis; however it required more memory. This was because in the non current mode, the spectrum of a test could be decoded and analysed. After analysis, the object would be nulled allowing for the garbage collector to free the memory. However, during concurrent execution, the decoded objects were only nulled after a full pipeline execution as another thread may be using that object. \todo{Not to sure if this is explained very well}

The next analysis implementation was the total difference. This method discarded the order of the methods for a test. Instead sorted them alphabetically and for every difference between the pairs, it added the 1 to the difference. This worked well in conjunction with the bloom filter approach, as it was a fast analysis pipeline while still returning high accuracy results. \todo{Should I explain what high accuracy is ?}

The next idea implemented was to take into account for the ‘calling context tree’. This means that for each method call, the data contains a separate node for each call stack that the method was called with \cite{callingcontext}. What this involves in java terms is using the stack trace of the method and storing the depth specified within the properties file. It is therefore more difficult for a test to be similar to another as the calling tree of a method call is used.

Finally, the last implemented method was weighting. The idea behind it was that there are some methods that will be contained within the majority of tests spectrum and occur frequently in the spectrum. These methods will often be low level, such that every test will execute them. Therefore giving less information about the level of redundancy than lower frequency method calls. So the idea in weighting is to increase the 'importance' of lower frequency method calls by discarding the top 20\% method calls.

\section{Results}
\todo{Should I compare the current analysis methods ? ie. Show more results such as the number of redundant tests each method gives ?}
The result that will be shown will be from the Whiley benchmark using the Levenshtein distance metric. From looking at the graphical representation \ref{fig:similiartests}, it shows the tests that are matching for Byte\_Valid\_5 test.

\begin{figure}[h]
\caption{GUI showing similar test cases}
\label{fig:similiartests}
\includegraphics[width=\textwidth]{model24.png}
\end{figure}

This requires us to conduct further examination of the test and as such the tests are from the Whiley Compiler tests, so looking at the Whiley files would give us insight into how closely related these two tests are. Comparing Byte\_Valid\_5 and Byte\_Valid\_8 in figures \ref{fig:byte8} and \ref{fig:byte5} respectively. It shows that the difference between the tests is the range of j and two additional addition statements. The decision whether these two tests are redundant is up to a developer. But there is a possibility that they could be separated or merged to create a single test case.

\begin{figure}[h]
\caption{Byte\_Valid\_8}
\begin{lstlisting}
public method main(System.Console sys) -> void:
    for i in constants:
        for j in 0 .. 8:
            sys.out.print(Any.toString(i) ++ " << ")
            sys.out.print("1+" ++ Any.toString(j) ++ " = ")
            sys.out.println(Any.toString(i << (1 + j)))
\end{lstlisting}
\label{fig:byte8}
\end{figure}

\begin{figure}[h]
\caption{Byte\_Valid\_5}
\begin{lstlisting}
public method main(System.Console sys) -> void:
    for i in constants:
        for j in 0 .. 9:
            sys.out.print(Any.toString(i) ++ " << ")
            sys.out.print(Any.toString(j) ++ " = ")
            sys.out.println(Any.toString(i << j))
\end{lstlisting}
\label{fig:byte5}
\end{figure}




\chapter{Results and Discussion}\label{C:results}\label{C:evaluation}

This section reports on the outcome of several experiments involving our tool on a realistic benchmark suit. These results are in regard to the time taken, number of comparisons and the types of redundant tests identified. A range of the techniques discussed in Chapter \ref{C:workdone} will be examined, as discussed in the next section. 

There are two points that need to be taken into account before reading the results. Firstly, in the significant tables below, a `+' represents that significant increase, `-' significant decrease and `=' represents no significant difference. Secondly, the graphs are displayed using a logarithm scale.
\section{Method}

This section will go through the process of how the data was retrieved, analysed and evaluated.

Firstly, the benchmarks had to be set up in such a way that the tests could be run locally. This involved retrieving the dependencies of each benchmark and setting up an build script through ant for each, in order to run the tests. After the tests had finished, the trace information was then saved to disk.The data was then run on a grid computing setup. This setup involves a large number of computers which execute the instructions given to them. Lastly, after the results had been returned, a Wilcoxon signed rank test was used to determine whether the results are significantly different or not. A Wilcoxon signed rank test is a paired difference test that compares two related samples. The significant level used is 95\%. A total of seven different property settings were run per benchmark. Each property setting was run 30 times. 
\paragraph{}
The aim of this paper is mentioned in Section \ref{C:intro}, the main two focus points is the creation of a framework to identify redundant test cases and experimenting the different techniques on realistic benchmarks. The framework allows users to identify potentially redundant test cases however, it also gives them a range of settings and variations to use. Friction may be created if no guidance is given to users of the framework in regard to what settings to use. To create this guidance, the rest of this section explores the experiments conducted.

\subsection{Experiment \rom{1} - Pipeline Length Comparison}

The motivation behind the experiment was to analyse how the different pipeline lengths impact the performance of the framework. There were two different pipeline size's tested, two and three respectively. The final pipeline stage was the same for both pipeline sizes explored. A single pipeline comparison was left out due to the amount of memory and time taken that was required for it to finish analysing.

\begin{itemize}
\item Difference between runs: Pipeline size
\end{itemize}

\subsection{Experiment \rom{2} - K Depth Comparison}

One of the abilities of the framework is to trace the K depth specified in the settings. This experiment is to determine the impact that altering the depth of K has on the performance. There are three K depth's explored, one, two and three respectively. These were chosen after consideration in regard \todo{Finish}

\begin{itemize}
\item Difference between runs: K Depth
\end{itemize}

\subsection{Experiment \rom{3} - Parameter Comparison}

The parameters of a method call is important information, this experiment explores the performance difference between the Pipeline 2 settings as used above, with the same settings and parameters used. \todo{Redo}

\begin{itemize}
\item Difference between runs: Parameters Used
\end{itemize}

\subsection{Experiment \rom{4} - Weighting Comparison}

Utilising weighting is an attempt to remove any false positives situations. The experiment attempts to identify if weighting has any potential to remove these false positives, while exploring the impact of performance.

\begin{itemize}
\item Difference between runs: Weighting Used
\end{itemize}

\subsection{Experiment \rom{5} - Weighting and Parameter Comparison}

Combining weighting and parameters is an attempt to reduce the number of false positives, and at the same time increasing the confidence of the redundant tests being redundant. 

\begin{itemize}
\item Difference between runs: Weighting and Parameters Used
\end{itemize}

\section{Environmental Methodologies}
\label{enviro}
As previously discussed in Section \ref{performanceEvalBG} there are a variety of challenges that present themselves when using Java to evaluate the performance of a particular application. These challenges are discussed throughout the section.

\subsection{Distributed Grid Computing}
 A distributed grid computing system was used to execute the data analysis. This involved utilising idle computers located around Victoria University of Wellington's School of Engineering and Computer Science. A total of 150 jobs can be executed concurrently. If a user was to log on while a process was being conducted, the application would be paused until the user left.

\subsection{Measuring Time}
One of the variables that is considered important for evaluating the performance of the tool is the time taken. In Java, this can be achieved by accessing a system method that returns the total time in milliseconds from 1970 00:00:00 UTC to the current time. This notation of time is defined as the wall clock time taken. Utilising a distributed grid system presents difficulties with using the wall time to measure the time taken. As mentioned in Section \ref{enviro} when a job is run on the grid system, there is potential for the computer the job is running on to be paused. If wall time were to be used, the time that the process is paused would also be included. To resolve this issue, the notation of CPU time was used, this is the measure of time in nanoseconds that the thread was being executed on the CPU. Secondly, if a user logs in during analysis, the RAM being used by the JVM could potentially be moved into virtual memory dependent on the amount of RAM that the user needed. Both these issues were limited by running the tests at times when users were unlikely to log in such as overnight.
\paragraph{}
When measuring the time taken to analyse, the start up cost was not taken into account. Running up to 150 jobs meant that a large number of applications would be attempting to access a single data file at a single time. This stress on the system introduces a large amount of non-deterministic behaviour. Removing the start up cost avoided this issue.

\subsection{Hardware Environment}
\subsubsection{Heap Size}

The heap is the location that the JVM uses to store objects that are produced during the execution of an application. The amount of heap that was allocated to the JVM was 6gb for every benchmark over every run and was set through the grid system by requiring at least 6gb of free RAM before running the process on that machine. This ensured that the machine's had enough free RAM.

\subsubsection{Platform}

By using a distributed grid system, the platform required had to be specified. The platform that was used was a Linux 4.0.5 64 bit system with 8gb of RAM. This was specified through the grid system such that every machine that was used to analyse the data met these requirements.

\subsection{Software Environment}
\todo{Number of VM invocations?}

A default garbage collection strategy was used. This is known as a concurrent-mark-sweep strategy which uses multiple threads to scan the heap, mark unused objects and recycle them \cite{oracle2015}. The approach allows for a high throughput but tends to use more CPU time than other strategies. This was deemed worth the trade off as memory was the bottleneck rather than CPU for the framework and increasing the throughput was more important.

\section{Benchmarks}
\label{S:bench}
Suitable benchmarks had to be found to test the different metrics on. They were located by looking at popular java framework's, Github repositories and David Pearce's personal projects. The benchmarks had to meet a criteria where they were Java based, had reasonable number of tests (40+) and were open source. Although there were over ten potential benchmarks, the ability to use them depended on their build process. If they used Maven, it was difficult to create a jar that contained the tests and often meant that the amount of effort needed to get a working benchmark was higher than the benefit from it. This eliminated several potential benchmarks and left the projects that are built with either Ant or Gradle. In Table \ref{large_test} are the large bench marks used. These involved bench marks that required more than \todo{TODO} number of comparisons. In Table \ref{small_test} are the smaller bench marks used. 
\paragraph{}
A variety of test types and sizes of the benchmarks were examined. A range of sizes were used in order to fully evaluate the framework. Although the framework and method execution details is more aimed toward larger test suites, it would be expected that they will give the same performance in relation to large test cases with the metrics and spectra's used. A range of test types were also chosen, David J Pearce's projects contain end to end tests which run through a whole module at once, while the others attempt to test units of code at a time. Having a range of test types gives a wider evaluation view and gives insight into potential situations where method details may not be accurate in identifying redundant test cases.
\paragraph{}
For the larger benchmarks that produced up to 100,000 method calls per test, they required a large amount of memory to store it while the tests were running. Without the use of a database, this limited the number of test cases that could be collected. Whiley and Ant were the benchmarks that did not have all there tests executed.  

\begin{table}[]
\centering
\caption{Large Test Suites}
\label{large_test}
\begin{tabular}{|l|l|l|l|}
\hline
{\bf Benchmark}       &  {\bf Number of Tests} & {\bf Type of tests} & {\bf Authors}   \\ \hline
Whiley - Wyc Valid         &       &    End to End      & David J Pearce          \\ \hline
Spring - Core   &       &    Unit Tests      & Community \\ \hline
Metric-x - Core &       &    Unit Tests      & Community \\ \hline
Jasm              &             &    End to End      & David J Pearce \\ \hline

\end{tabular}
\end{table}

\begin{table}[]
\centering

\label{small_test}
\caption{Small Test Suites}
\begin{tabular}{|l|l|l|l|}
\hline
{\bf Benchmark}   & {\bf Number of Tests} & {\bf Type of tests} & {\bf Authors}  \\ \hline
Ant             &       &    Unit Tests      & Community \\ \hline
Imcache &           &    End to End        & Community \\ \hline
\end{tabular}
\end{table}

\section{Experiment \rom{1} - Pipeline Length Comparison}
\label{sec:pipelineEva}
The significant table for comparing the use of pipeline of size two vs size three is shown in Table \ref{pipelinesig}. It shows that there was a mixture of results for the total time taken to analyse the data. Metrics-x, Ant, Spring and Jasm performed significantly better with a pipeline of size 3 in comparison to size two. Imcache had no significant difference and Whiley performed significantly worse with a pipeline of size three in comparison to size two. Every benchmark has no significant difference between the number of redundant tests identified, every benchmark produced the exact same number of redundant tests in both pipeline size of two and three.
\paragraph{}
A chart showing how each benchmark reacted to the change in the pipeline size is shown in Figure \ref{fig:pipelinegraph}. It shows the number of test case comparisons that the final stage of the pipeline has to conduct.

\begin{table}[h]
\centering


\begin{tabular}{|l|l|l|}
\hline
{\bf }          & {\bf Total Time} & {\bf Redundant Tests Identified} \\ \hline
{\bf Whiley}    & +                & =                           \\ \hline
{\bf Jasm}      & -                & =                           \\ \hline
{\bf Ant}       & -                & =                           \\ \hline
{\bf Spring}    & -                & =                           \\ \hline
{\bf Imcache}   & =                & =                           \\ \hline
{\bf Metrics-x} & -                & =                           \\ \hline
\end{tabular}
\caption{A table showing the significant relationship between the use of pipeline of size two with pipeline of size three for each benchmark}
\label{pipelinesig}
\end{table}

\begin{figure}[h]
\begin{center}
\includegraphics[height=10cm, width = 14.5cm]{Pipeline.png}
\end{center}
\caption{A figure showing the effect that using pipelines has on the number of comparisons that the final stage (Most computationally heavy) has to do.}
\label{fig:pipelinegraph}
\end{figure}



\begin{itemize}
\item Increasing the pipeline increases the performance of the framework
\item No difference between the output redundant test cases
\end{itemize}

An interesting result was the impact of the length the pipeline had on the total time taken to analyse the data. The idea of the pipeline was to decrease the amount of comparisons that the each stage in the pipeline has to do. This in turn meant less memory consumption and time taken for the more computationally heavy stages. Looking at the results show that in most cases, having two pipeline stages meant a significantly decreased total time taken to analyse over a pipeline with three stages. Intuitively, this may go against what would make sense. Looking at Figure \ref{fig:pipelinegraph}, it shows that there is a major decrease in the number of comparisons that the last stage does between one stage and two or three stages. However, between two and three stages there is limited differences. These results imply that the test cases that took less time using a two stage, were spending more time on the second stage than they were saving from the reduced number of comparisons during the third stage. 
\paragraph{}
There would be two different reasons that this result may occur. Firstly, each benchmark had it's own level of redundancy that was being looked for, this was dependent on the size of the test cases of each. As benchmarks with smaller test cases needed to have a higher level of redundancy in comparison to benchmarks with larger test cases, as this meant they would have roughly the same percentage of difference. Since each benchmark had their own percentage, this implies that each benchmark also has its optimal settings for the second pipeline. The settings were chosen by experimenting with different options. Therefore, not using a near optimal second pipeline could have contributed to it. \todo{Need to explain it better}
\paragraph{}
The second reason is that there may be little difference between the second pipeline and third pipeline. For example, if the first pipeline reduces the number of comparisons needed to 50, then there is a limited number of comparisons that a second pipeline can remove. This leads to a situation where the number of identified redundant test cases is similar for both pipelines.
\paragraph{}
As expected of the redundant tests identified, for each benchmark there was no significant change. If there were a change between the two pipelines with the same final stage settings could imply two things. Firstly, there is a bug in the code. Secondly, the second pipeline stage is more specific than the last. Since both the second stage and third stage share the same final stage, this would cause the last stage to be redundant and the redundant tests different to that of the two stage pipeline. \todo{@DJ, Does this make sense?}


\section{Experiment \rom{2} - K Depth Comparison}
Figure \ref{fig:kdepthgraph} shows the change that the KDepth had on the number of redundant test cases identified. The amount of change is limited between one, two and three depth. Ant, Whiley and Spring appear to be the most responsive. \todo{Should I do sig test?}

\begin{figure}[h]
\begin{center}
\includegraphics[height=10cm, width = 14.5cm]{KDepth.png}
\end{center}
\caption{A figure showing the effect that a change in the depth of the calling context has on the number of redundant tests are identified.}
\label{fig:kdepthgraph}
\end{figure}

\begin{itemize}
\item Increasing the K Depth has a large impact on the identified tests
\item The extra computation is worth the trade off for the increased confidence
\end{itemize}

Increasing the depth of the calling context means adding data to the analysis stages, intuitively making it more difficult for test cases to be the same. This should mean there is a decrease in the number of comparisons as the calling context depth increases. It also means that more data is going to have to be looked at which implies an increase in time as calling context depth increases.
\paragraph{}
Examining Figure \ref{fig:kdepthgraph}, it shows that the number of redundant tests is only slightly decreased as the K depth increases. This may show several things. Firstly, it may imply that when no other settings such as weighting or parameters are used, then the depth has limited effect on the redundant test cases identified. This can be seen how there is a limited amount of change for each of the benchmarks as the K increases. Secondly, the number of comparisons that were output from the first pipeline stage may cause there to be a limited differentiation. If the first pipeline stage output a low number of comparisons, then the comparisons output would more likely to be similar regardless of the K depth specified. This is due to the amount of information that the analysis stage has access to. If two test cases are practically the same method calls, except for the parameters passed to methods, then the K depth will not pick this difference up.
\paragraph{}
Overall the results imply that for some projects the calling context does not have to be particularly deep when there are no other settings used. Section \ref{sec:param} and Section \ref{sec:weight} explore the impact that using parameters has on the output.

\todo{Maybe look at a coding for the different types of tests identified. What types does lower K mean more likely to pick up ?}

\section{Experiment \rom{3} - Parameter Comparison}
\label{sec:param}

The significant table for comparing the use of parameters is shown in Table \ref{parametersig}. It shows that there was a negative relation for every benchmark in regard to time taken, this shows that parameters caused an increase in the time taken to analyse the data. Every benchmark had a significantly positive effect on the number of redundant test cases identified, showing that parameters decrease the number of redundant test cases. This is not the case in Imcache due to it having 0 redundant test cases identified in both, therefore no difference between the two.


\begin{table}[h]
\centering


\begin{tabular}{|l|l|l|}
\hline
{\bf }          & {\bf Total Time} & {\bf Redundant Tests Identified} \\ \hline
{\bf Whiley}    & +                & -                           \\ \hline
{\bf Jasm}      & +               & -                          \\ \hline
{\bf Ant}       & +                & -                           \\ \hline
{\bf Spring}    & +                & -                           \\ \hline
{\bf Imcache}   & +                & =                           \\ \hline
{\bf Metrics-x} & +                & -                           \\ \hline
\end{tabular}
\caption{A table showing the significant relationship between the use of parameters and no parameters for each benchmark}
\label{parametersig}
\end{table}

\begin{figure}[H]
\begin{center}
\includegraphics[height=10cm, width = 14.5cm]{Parameters.png}
\end{center}
\caption{A figure showing the effect that using parameters has on the number of redundant tests are identified.}
\label{fig:paramgraph}
\end{figure}


\begin{itemize}
\item Increases the time taken
\item Decreases the tests identified
\end{itemize}

\todo{Appears that parameters may increase the non-deterministic attirbutes of java as mentioned.}
Intuitively, recording parameters of method calls would increase the time taken to calculate the level of redundancy due to the extra amount of data generated. However, parameters only add extra computation when the method execution of the given K is exactly the same. For example, if a test execution is $A ->  B ->  C$ with D Parameter is compared to $A ->  B ->  F with E Parameter. Then the parameter won't be taken into account and therefore add limited amount o$f time. Another method which would cause time to be added is the amount of time that the VM spends garbage collecting, due to increased amount of information in memory, the garbage collection process will have to free up more memory for the analysis to continue to run.
\paragraph{}
A larger calling context would mean that it would be more difficult for two tests to match. Matching with a K value of three would be expected to be a rare occurrence in the scale of a benchmark. The parameters can be large and hold a lot of information therefore if the K value matches, then matching parameters may incur a large cost and would be expected to have limited matching. Taking into account this extra information, it would make sense that the number of redundant tests identified would decrease, but the total time taken would increase. 
\paragraph{}
Looking at the significant Table \ref{fig:paramgraph}, this matches what is expected to occur. For every benchmark, using parameters significantly increases the time taken. This may imply that the number of tests that match the K value of three was higher than expected, causing more computations to look at the parameter information. Examining Appendix \ref{fig:paramtime} shows how some benchmarks react differently to benchmarks. The most interesting would be how Jasm reacts to the use of parameters. Without parameters, it analyses the data in less than one minute, with parameters, it takes just under five minutes. One reason this may occur is the number of tests that match K value is higher than average, without weighting the set up methods are taken into account. If these set up methods are a large majority of the method calls, and match each other for the K value and parameters, this would increase the time taken substantially. Examining the weighting and parameter time taken in Appendix \ref{fig:weightparamtime}, it shows that when weighting is taken into account with parameters then for Jasm, the time taken goes from just under five minutes to just under one minute.
\paragraph{}
Every benchmark caused a significant decrease in the number of redundant test cases. This backs up the implication that there were more situations where a K depth of three matched than expected meaning that parameters had a significant effect on the redundant tests identified. Looking at Figure \ref{fig:paramgraph} shows that every benchmark reacted with a substantial decrease in redundant tests identified, with Whiley being the least affected. 

\todo{Test on Calling context = 2 ? Look at significant differences}


\section{Experiment \rom{4} - Weighting Comparison}
\label{sec:weight}
The significant table for comparing the use of weighting is shown in Table \ref{weightingsig}. There are a mix for the benchmarks in regard to the total time taken to analyse the data. Whiley, Ant and Imcache had a significantly negative relation and weighting increased the time taken to analyse. Jasm, Spring and Metric-x had a significantly positive relation and weighting decreased the time taken to analyse. Whiley, Ant, Jasm and Metrics-x had a significantly positive relation in regard to the number of redundant tests identified, showing a decrease in the number identified when weighting was applied. Spring was the only benchmark where weighting had a significantly negative impact, increasing the tests identified.

\begin{table}[h]
\centering


\begin{tabular}{|l|l|l|}
\hline
{\bf }          & {\bf Total Time} & {\bf Redundant Tests Identified} \\ \hline
{\bf Whiley}    & +                & -                           \\ \hline
{\bf Jasm}      & -                & -                           \\ \hline
{\bf Ant}       & +                & -                           \\ \hline
{\bf Spring}    & -                & +                           \\ \hline
{\bf Imcache}   & +                & =                           \\ \hline
{\bf Metrics-x} & -                & -                           \\ \hline
\end{tabular}
\caption{A table showing the significant relationship between the use of weighting and no weighting for each benchmark}
\label{weightingsig}
\end{table}


\begin{figure}[h]
\begin{center}
\includegraphics[height=10cm, width = 14.5cm]{Weighting.png}
\end{center}
\caption{A figure showing the effect that using weighting has on the number of redundant tests are identified.}
\label{fig:weightgraph}
\end{figure}


\begin{itemize}
\item Improves the false positive rates
\item Decreases the time taken
\end{itemize}


\todo{Weighting appears to reduce the variance in the time taken}
As previously discussed in \todo{section 2 and 3} many of the other papers encountered difficulties when test cases shared setup and teardown method calls, meaning that some of the methods picked up caused a high false - positive rate. Intuitively this makes sense when the test cases are small or the setup and teardown is large, meaning the setup and teardown methods make up a larger majority of the test calls. By taking the approach of removing a portion of the most executed method calls meant that the amount of data that would be compared decreases. This should reflect onto the results by showing a decrease in the time taken. The effect that it has on the number of redundant test cases should be dependent on the benchmark when weighting is used. This is because removing the most common tests should imply an decreased false-positive rate, but also may increase the number of actual redundant tests picked up. Therefore, depending on which on the two variables outweighs the other, will result in an increase or decrease in the number of redundant test cases identified.
\paragraph{}
Looking at the results in Table \ref{weightingsig}, it shows there was a mixture of results in regard to the total time taken. Two out of the three benchmarks that significantly increased in total time were from the small benchmarks. This may imply that the size of the benchmark has some relation to the effect of weighting on the time taken. The reason is, weighting is calculated once for each test case at the start of the pipeline stage. This weighting calculation can take a varying length of time, depending on the number of test cases. Since the weighting is linear in relation to the number of test cases, however, the comparisons is factorial in relation to the test cases. This shows that calculating the weighting for smaller benchmarks may take more time than it saves. However, the goal of weighting is not to save time, but it is to reduce the number of false positives.
\paragraph{}
Spring was the only benchmark that significantly increased the number of redundant test cases identified. The other benchmarks had a significant decrease. At first, this makes it appear that by using weighting it may solve some of the issues that were identified in \cite{koochakzadeh2009test} \cite{li2008static}. To confirm this, examining Table \ref{whileycoding} and \ref{metriccoding} will give insight into the effect of weighting. Comparing the two columns, Pipeline 2 and Weighting it shows the only difference is a removal of the two limited redundancy test cases that were picked up by Pipeline 2. The weighting is able to remove the test cases that have similar set up and tear down methods, but fails to reduce the number of other redundancies. By removing the method executions that were common throughout the benchmarks, this lead to a decrease in false-positive tests identified.


\section{Experiment \rom{5} - Weighting and Parameter Comparison}

The significant table for comparing the use of weighting with parameters is shown in Table \ref{weightingparamsig}. It shows that for every Benchmark apart from Imcache, there was a negative relation for the time taken, showing that the time increased when using weighting and parameters for the majority of benchmarks. In contrast to this, every benchmark had a positive relation to the number of redundant tests identified.

\begin{table}[h]
\centering

\begin{tabular}{|l|l|l|}
\hline
{\bf }          & {\bf Total Time} & {\bf Redundant Tests Identified} \\ \hline
{\bf Whiley}    & +                & -                           \\ \hline
{\bf Jasm}      & +                & -                           \\ \hline
{\bf Ant}       & +                & -                           \\ \hline
{\bf Spring}    & +                & -                           \\ \hline
{\bf Imcache}   & -                & =                           \\ \hline
{\bf Metrics-x} & +                & -                           \\ \hline
\end{tabular}
\caption{A table showing the significant relationship between the use of weighting with parameters and neither for each benchmark}
\label{weightingparamsig}
\end{table}

\begin{figure}[h]
\begin{center}
\includegraphics[height=10cm, width = 14.5cm]{WeightNParam.png}
\end{center}
\caption{A figure showing the effect that using weighting and parameters has on the number of redundant tests are identified.}
\label{fig:weightingparamgraph}
\end{figure}

\begin{itemize}
\item Improves the false positive rates
\item Decreases the time taken
\end{itemize}

\todo{More}
The result from a combination of both, weighting and parameter provides an interesting set of results. Looking at Table \ref{weightingparamsig}, it shows that it is nearly identical to the parameters table with the only exception being that the Imcache benchmark changed from significant increase in time to a significant decrease in time. This indicates that parameters may have a stronger effect on the final outcome in comparison to weighting. This is an unexpected result. As previously discussed in Section \ref{sec:param}, one of the reasons that parameters increase the time taken is due to the similar setup and tear down method calls, causing the number of k depth matches to increase and the parameters to be examined more often. With weighting removing a large portion of the common method executions it would be expected that the portion of K depth matches decreases in relation to the number of method calls, causing a decrease in time between weighting and parameters combined and parameters. Looking at Figure \ref{fig:weightparamvparamtime}, this appears to be true for Jasm, Metric-x, Spring and Imcache. However, for Whiley and Ant benchmarks the time to calculate the weighting information is taking longer than the comparisons saved. 



\section{Redundant Test Case Coding}

Table \ref{whileycoding} and Table \ref{metriccoding} compare the four major variations of techniques used to show the different types of redundancy that were picked up for the Whiley and Metric-x benchmarks respectively. The tables show a list of codings, these codings are the types of redundancy that the framework identified. It is important to note that the numbers represent matching test cases, not redundant pairings. Such that there are 5 pairs that are the same, but 10 tests that match \todo{A bit counter intuitive, might need to redo to be pairing?}.
\paragraph{}
Table \ref{whileycoding} shows that the use of weighting removed the two test cases that had no redundancy, parameters removed the same as weighting as well as the different array value test cases and parameters and weighting combined matched the parameters. Table \ref{metriccoding} \todo{describe metric coding table}

\begin{table}[h]
\centering

\begin{tabular}{|l|l|l|l|l|}
\hline
                          & \multicolumn{4}{c|}{{\bf Whiley}}                                                             \\ \hline
{\bf Types of redundancy} & \multicolumn{1}{c|}{{\bf Pipeline 2}} & {\bf Weighting} & {\bf Parameters} & {\bf Parameters and Weighting} \\ \hline
Different Equation Value  & 6                                     & 6               & 6                & 6                \\ \hline
Different Equation Sign   & 8                                     & 8               & 8                & 8                \\ \hline
Different Array Values    & 2                                     & 2               & 0                & 0                \\ \hline
Same                      & 10                                    & 10              & 10               & 10               \\ \hline
Limited Redundancy        & 2                                     & 0               & 0                & 0                \\ \hline
Rearranged Equation       & 2                                     & 2               & 2                & 2                \\ \hline
Extra if statement        & 2                                     & 2               & 2                & 2                \\ \hline
                          &                                       &                 &                  &                  \\ \hline
{\bf Total}               & 32                                    & 30              & 28               & 28               \\ \hline
\end{tabular}
\caption{A table displaying a list of coding's for the Whiley Benchmark for four of the different techniques used.}
\label{whileycoding}
\end{table}

\begin{table}[h]
\centering

\begin{tabular}{|l|l|l|l|}
\hline
                          & \multicolumn{3}{c|}{{\bf Metric-X}}                   \\ \hline
{\bf Types of redundancy} & {\bf Weighting} & {\bf Parameters} & {\bf Parameters and Weighting} \\ \hline
Different Parameter Value & 10              & 8                & 4                \\ \hline
Different Object Type     & 2               & 2                & 0                \\ \hline
Different Array Values    & 6               & 4                & 6                \\ \hline
Similar                      & 2               & 0                & 0                \\ \hline
Limited Redundancy        & 4               & 6                & 0                \\ \hline
                          &                 &                  &                  \\ \hline
{\bf Total}               & 24              & 20               & 10               \\ \hline
\end{tabular}
\caption{A table displaying a list of coding's for the Whiley Benchmark for four of the different techniques used.}
\label{metriccoding}
\end{table}
\chapter{Future Work and Conclusions\improvement{Dave}}\label{C:future}

This chapter will first discuss potential future work which would increase the usefulness of the tool for a wider range of audiences. The experiments are then reflected on and finally concluded. \todo{Finish properly}

\section{Future Work}
The tool created throughout the project allows for the trace information from a test suite to be stored and analysed using different techniques. The information we explored using to determine redundant tests was the method calls. Throughout the project, three main areas have been identified for future work.

\begin{itemize}

\item \textbf{Handle larger test suites.} The main objective would be to increase the ability of the tool to handle larger test suites. Currently, the tool relies on RAM to hold the trace and analysis information however, the amount of data held can often grow rapidly if the test cases contain a large number of method calls. An approach to get around this would be to integrate the tool with a database. This would allow the tool to handle arbitrary large suites. The down side would the speed of the tool would decrease as the read/write to disk is slower than to RAM. Giving developers this option would be preferred. 

\item \textbf{Trace statement information.} The tool has the ability to trace method call information and has potential to be used on large test suites. This approach may not be the best when we are only looking at smaller test suites. Another approach that could be implemented is tracing the statement coverage information. This would allow us to compare the statement and method coverage information directly and give us more insight into the issue of identifying redundant tests. Tracing the statement information would allow us to further explore the use case mentioned in Chapter \ref{C:intro}. The use case was the ability to split the test suites into two, one containing the non redundant test cases and the other was the original suite. This could be further expanded with statement coverage information. By identifying when a test subsumes another, the test subsumed could be moved into another suite. If the larger test fails, the subsumed tests could then be ran to help pin point the error in the code.

\item \textbf{Combining Statement and method information.} The method call information was a good heuristic to use. On the other hand, it is difficult to judge how good method calls are overall at identifying redundant test cases without any comparison to statement coverage or purposeful redundancy and bugs added. One application could be to combine the statement and method information together. Method information could be used as a heuristic and statement coverage could then explore the test cases identified by the heuristic.

\end{itemize}

\section{Conclusions}

There were two main contributions of the paper as discussed in Chapter \ref{C:intro}:

\begin{itemize}
\item Create a tool for identifying redundant test cases (Chapter 3)
\item Analyse different strategies for identifying redundant test cases through experimenting on realistic benchmarks (Chapter 4)
\end{itemize}

Throughout the report, the tool has been discussed along with the different techniques implemented. The tool first had to trace the data. This was achieved through the Aspectj framework. After the test suites were able to be traced, the trace information then had to be filtered with different spectra. The ability to filter out information was one of the key features of the tool. This allowed developers to trace information, then run a range of different filters over the data without being constrained to one spectra. The three spectra filters were ``unique method calls", ``all method calls" and "call tree". The other key feature of the tool was the pipeline. Integrating it with the spectra filters allowed for the "unique method calls" to be used as a heuristic to reduce the number of comparisons that the subsequent stages had to perform. The tool then was altered to trace the parameter information from the test cases. This was achieved by using reflection on the parameter objects. Finally, the ability to apply a weighting was implemented. This was achieved by removing the top 20\% method calls.

\paragraph{Method Call Information}
The use of method call information was a good candidate to identify redundant tests. They allow the tool to look at test suites that produce a larger amount of total information, in particular ones which are end to end tests -- Whiley and Jasm. As mentioned in the Future Work section above and discussed in Experiment \rom{2}, the method calls appeared to be particularly good at being a heuristic. Filtering with the "unique method calls" spectra, the tool was able to remove a large portion of the test case comparisons in the final stage had to perform.

\paragraph{Pipeline}
Pipelining was one of the critical aspects of the project. It not only lead to a decrease in the time taken but also the memory consumed. An important thing to note is each benchmark had an optimal length of the pipeline, using a length larger caused more time to be taken to analyse than was saved. Pipelining particularly worked well with the use of a heuristic filter.

\paragraph{Call tree depth}
The experiments showed that as the depth of the call tree increased, there was limited level of reduction in the number of redundant tests identified. This result was unexpected. Although there was a change, it was lower that what was expected. This implies that by increasing the K depth without any other technique, there is limited improvement. 

\paragraph{Parameter}
By using reflection to retrieve the fields of these objects it allowed more insight into the content that the parameter objects were holding and the nature of them, but it also meant more data had to be held and analysed resulting in an increased time taken. There exists this trade off between time taken and confidence of redundancy when using parameters. 

\paragraph{Weighting}
The weighting technique implemented attempts to improve the accuracy of the tool by removing false-positives while improving the time taken to analyse. Experiment \rom{4} discussed how weighting was able to reduce the level of false-positives but needs further improvement.

\subsection{Conclusion}

The project has explored the use of higher level information than previously explored for identifying redundant test cases while examining how different techniques can impact the results. The output of the tool is adequate in the sense that there is a low number of tests identified and these tests contain limited non-redundant tests. 


The results would be expected to build off those achieved in Maurer et al. \cite{li2008static}  and Robinson et al. \cite{koochakzadeh2009test}. 



\begin{appendices}
\chapter{The effect of pipeline size in regard to the time taken}
\begin{figure}[h]
\centering
\includegraphics[width=\textwidth,height=13cm]{PipelineTime.png}
\caption{A figure showing the relationship that using different pipeline sizes has on the total time taken to analyse the data.}
\label{fig:pipelinetime}
\end{figure}

\chapter{The effect of parameters in regard to the time taken}
\begin{figure}[h]
\centering
\includegraphics[width=\textwidth,height=13cm]{ParamTime.png}
\caption{A figure showing the relationship that using parameters has on the total time taken to analyse the data.}
\label{fig:paramtime}
\end{figure}

\chapter{The effect of weighting in regard to the time taken}
\begin{figure}[h]
\centering
\includegraphics[width=\textwidth,height=13cm]{WeightTime.png}
\caption{A figure showing the relationship that using weighting has on the total time taken to analyse the data.}
\label{fig:weighttime}
\end{figure}

\chapter{The effect of weighting and parameters combined in regard to the time taken}
\begin{figure}[h]
\centering
\includegraphics[width=\textwidth,height=13cm]{WeightnParamTime.png}
\caption{A figure showing the relationship that using weighting and parameters combined has on the total time taken to analyse the data.}
\label{fig:weightparamtime}
\end{figure}

\chapter{The effect of weighting and parameters vs parameters in regard to the time taken}
\begin{figure}[h]
\centering
\includegraphics[width=\textwidth,height=13cm]{WeightnParamvParamTime.png}
\caption{A figure showing the relationship that using weighting and parameters combined vs parameters alone has on the total time taken to analyse the data.}
\label{fig:weightparamvparamtime}
\end{figure}




\end{appendices}

%%%%%%%%%%%%%%%%%%%%%%%%%%%%%%%%%%%%%%%%%%%%%%%%%%%%%%%

\backmatter

%%%%%%%%%%%%%%%%%%%%%%%%%%%%%%%%%%%%%%%%%%%%%%%%%%%%%%%


%\bibliographystyle{ieeetr}
\bibliographystyle{acm}
\bibliography{sample.bib}


\end{document}
